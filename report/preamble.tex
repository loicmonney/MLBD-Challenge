\usepackage[francais]{babel}

% Use UTF-8 for plain tex files
\usepackage[utf8]{inputenc}

% Datetime for \today command
\usepackage{datetime}

% Advanced math support
\usepackage{amsmath}
\usepackage{mathabx}
\usepackage{xfrac}
\usepackage[autolanguage]{numprint}

% Advanced citation tools
\usepackage[numbers,sort&compress]{natbib}

% Support for 2nd, 3rd,...
\usepackage[super]{nth}

% Set the margin from the page side
\usepackage[margin=3cm, left=3.5cm, right=3.5cm, twoside=false, marginparwidth=2.5cm, marginparsep=5mm]{geometry}
\usepackage{lscape}
\usepackage{pdflscape}
\usepackage[strict]{changepage}

% Call original TeX \input to avoid conflicts
\makeatletter
\newcommand*\ExpandableInput[1]{\@@input#1 }
\makeatother

\newcommand\includescript[1]{\ExpandableInput{|"#1"}}

% Support for internal links
\usepackage{hyperref}
\hypersetup{
    colorlinks=true,
    linkcolor=black,
    citecolor=black,
    filecolor=black,
    urlcolor=black,
}

% Typeset paragraph titles
\usepackage[noadjust]{marginnote}
\usepackage{setspace}
\newcommand{\p}[1]{%
	\leavevmode\marginnote{\vspace{-.8mm}\sffamily\scriptsize\setstretch{1.4}#1}%
	\ignorespaces
}

% Inline lists
\usepackage{paralist}

% Don't indent paragrpahs, instead separate them
\usepackage{parskip}
\setlength{\parskip}{5mm plus2mm minus3mm}
\setlength{\parindent}{0cm}

% Use alternative font (see http://www.tug.dk/FontCatalogue/ for alternatives)
\usepackage{cmbright}
\renewcommand\familydefault{\sfdefault} % Set the default font to be sans-serif
\usepackage[T1]{fontenc}
\usepackage{textcomp}
\usepackage{MnSymbol}

% Set line height multiplicity
\linespread{1.05}

% Allow URL typesetting
\usepackage{url}

% Allow email typesetting
\newcommand{\email}[1]{%
	\href{mailto:#1}{\texttt{#1}}%
}

% Allow advanced list typesetting
\usepackage{enumitem}

% Customize captions appearances
\usepackage[justification=centering]{caption}

% Customize headers and footers
\usepackage{fancyhdr}

% Includegraphics support
\usepackage{graphicx}

% Allows to change text color
\usepackage[usenames,dvipsnames,table]{xcolor}

% Assign the LastPage label to the last page
\usepackage{lastpage}

% Help better structure documents
\newcommand{\content}{%
	\cleardoublepage%
	\setcounter{page}{1}
}


% Enable the creation of appendices
\usepackage{appendix}

% Enable the creation of glossaries
%%\usepackage[nomain,acronym]{glossaries}
%%\usepackage{glossary-inline}
%%\renewcommand*{\glsautoprefix}{glo:}
%%\renewcommand*{\glspostdescription}{ }

%\newglossary{main}{0.gls}{0.glo}{Glossary}
%%\newglossary{concepts/general}{1.gls}{1.glo}{General concepts}
%%\newglossary{concepts/classification}{2.gls}{2.glo}{Graph classifications}
%%\newglossary{concepts/geometry}{3.gls}{3.glo}{Geometry concepts}

%%\input{concepts.gls}
%%\input{glossary.gls}
%%\makeglossaries

% Support for lists
\usepackage{etoolbox}
\usepackage{fp}

% Advanced templating support
\newcommand{\settitle}[2]{%
	\title{#1 -- #2}
	\def\doctitle{#1}
	\def\subtitle{#2}
}
\newcounter{mseaindex}
\newcounter{mseacount}
\newcommand{\addauthor}[1]{%
	\listadd{\authors}{#1}
	\stepcounter{mseacount}
}
\newcommand{\fmtauthor}[1]{#1}
\newcommand{\lineslist}[1]{#1\\}
\newcommand{\authorlines}{%
	\setcounter{mseaindex}{0}%
	\FPdiv{\limit}{\arabic{mseacount}}{2}%
	\FPround{\limit}{\limit}{0}%
	\renewcommand*{\do}[1]{%
		\ifnumequal{\value{mseaindex}}{0}{}{\ifnumequal{\value{mseaindex}}{\limit}{}{, }}%
		\ifnumequal{\value{mseaindex}}{\limit}{\\[0.7mm]\fmtauthor{##1}}{\fmtauthor{##1}}%
		\stepcounter{mseaindex}%
	}%
	\dolistloop{\authors}%
}
\newcommand{\authorslist}{%
	\setcounter{mseaindex}{0}%
	\renewcommand*{\do}[1]{%
		\ifnumequal{\value{mseaindex}}{0}{}{, }%
		\fmtauthor{##1}%
		\stepcounter{mseaindex}%
	}%
	\dolistloop{\authors}%
}

% Support for nomenclature
\usepackage{nomencl}

\makeatletter
\def\thenomenclature{%
  \setlength{\nomitemsep}{-0.8\parsep}
  \vspace{\parskip}\hrule
  \nompreamble
  \list{}{%
    \labelwidth\nom@tempdim
    \leftmargin\labelwidth
    \advance\leftmargin\labelsep
    \itemsep\nomitemsep
    \let\makelabel\nomlabel}}
\def\endthenomenclature{%
  \endlist
  \nompostamble
  \vspace{\parskip}\hrule}
\makeatother

\makenomenclature

% Set layout lengths
\setlength{\headheight}{8mm}
\setlength{\footskip}{1.5cm}
\addtolength{\textheight}{-.5cm}

% Set titles whitespace
\usepackage[explicit]{titlesec}
\titlespacing{\chapter}{-5mm}{8mm}{3mm}
\titlespacing{\section}{-5mm}{3mm}{-2mm}
\titlespacing{\subsection}{-5mm}{2mm}{-2mm}
\titlespacing{\subsubsection}{-5mm}{2mm}{-1mm}

\titleformat{\chapter}{\LARGE\bfseries}{\llap{\thechapter}}{0mm}{%
    \hfill\begin{minipage}[t]{\dimexpr\textwidth}\raggedright#1\end{minipage}%
}
\titleformat{\section}{\large\bfseries}{\llap{\thesection}}{0mm}{%
    \hfill\begin{minipage}[t]{\dimexpr\textwidth}\raggedright#1\end{minipage}%
}
\titleformat{\subsection}{\bfseries}{\llap{\thesubsection}}{0mm}{%
    \hfill\begin{minipage}[t]{\dimexpr\textwidth}\raggedright#1\end{minipage}%
}
\titleformat{\subsubsection}{\bfseries}{\llap{\thesubsubsection}}{0mm}{%
    \hfill\begin{minipage}[t]{\dimexpr\textwidth}\raggedright#1\end{minipage}%
}

% Enable code listings and set default options
\usepackage{listings}
\usepackage{courier} % Use Adobe Courier instead of Computer Modern Typewriter
                     % for monospaced text

\usepackage{hyphenat}

\newsavebox\leftarrowbox
\sbox\leftarrowbox{\raisebox{-.65mm}{\hspace*{0.3mm}\footnotesize\textcolor{gray}{$\rhookleftarrow$}}}
\makeatletter
\renewcommand*{\BreakableSlash}{%
  \leavevmode
  \prw@zbreak
  \discretionary{\usebox\leftarrowbox}{}{}%
  \prw@zbreak
}
\makeatother

\lstloadlanguages{Python,bash,sh,xml}
\lstset{
	language=Python,
	basicstyle=\footnotesize\ttfamily,
	stringstyle=\color{OliveGreen},
	numbers=left,
	numberstyle=\color{gray}\tiny,
	commentstyle=\color{magenta},
	keywordstyle=\color{MidnightBlue}\bfseries,
	frame=tb,
	rulecolor=\color{black},
	numbersep=5pt,
	escapechar=¶,
	extendedchars=true,
	captionpos=t,
	breaklines=true,
	showspaces=false,
	showtabs=false,
	tabsize=4, 
	xleftmargin=0pt,
	framexleftmargin=0pt,
	framexrightmargin=0pt,
	framexbottommargin=0pt,
	showstringspaces=false,
	belowcaptionskip=2mm,
}

%%\input{languages/diff.tex}  % Diff files highlighting
%%\input{languages/prompts.tex}  % Special formatting for command prompts
%%\input{languages/javascript.tex}  % Javascript highlighting 

\newcommand{\lref}[1]{%
	line~\ref{l:#1}%
}


% Enable the use of footnotes in section headings
% http://www.tex.ac.uk/cgi-bin/texfaq2html?label=ftnsect
\usepackage[stable]{footmisc}

% Limit the TOC dept to sections and subsections
\setcounter{tocdepth}{1}
\setcounter{secnumdepth}{3}

\usepackage{multicol}
\usepackage{multirow} % Enable columns spawning multiple rows
\usepackage{tabularx}
\usepackage{booktabs} % Provides {top|middle|bottom}rule
\usepackage{longtable} % Support for tables spawning multiple pages,
                       % captions and footnotes.
                       % Does not work in multicolumn environments
                       
% Commande pour les cellules d'en-tête de tableau
% Accèpte 1 paramètre : le contenu de la cellule
\newcommand{\thead}[1]{ %
	{\cellcolor[gray]{0.9} % Applique une coloration à la cellule
	\textbf{#1} % Ajoute le texte en gras (arg 1)
	}
}

% Define new tabularx column types:
%  - R: streteched right aligned
%  - C: stretched centered
\newcolumntype{R}{>{\raggedleft\arraybackslash}X}%
\newcolumntype{C}{>{\centering\arraybackslash}X}%
\renewcommand{\arraystretch}{1.3} % Set row height multiplicator
\usepackage{threeparttable} % Manual footnotes counter

% Provide support for (indented) unnumbered chapter and section titles

\def\cleartoodd{%
  \clearpage%
  \ifodd\value{page}\else\mbox{}\thispagestyle{empty}\newpage\fi%
}

\def\clearchap{%
  \ifodd\value{page}\else\mbox{}\thispagestyle{empty}\fi%
}

\let\origdoublepage\cleardoublepage
\renewcommand{\cleardoublepage}{%
  \cleartoodd%
}


\usepackage{tocloft}
\newcommand{\uchapter}[1]{
  \cleartoodd
  \chapter*{#1}
  \thispagestyle{chapterstart}
  \addcontentsline{toc}{chapter}{\hspace{\cftchapnumwidth}#1}
}
\newcommand{\usection}[1]{
  \phantomsection
  \section*{#1}
  \addcontentsline{toc}{section}{\hspace{\cftsecnumwidth}#1}
}
\newcommand{\usubsection}[1]{
	\subsection*{#1}
	\addcontentsline{toc}{subsection}{\hspace{\cftsubsecnumwidth}#1}
}
\newcommand{\lchapter}[2][\_nolabel]{
	\cleartoodd
	\chapter{#2}
	\thispagestyle{chapterstart}
	\ifthenelse{\equal{#1}{\_nolabel}}{}{\label{sec:#1}}
}


\let\origchapter\chapter
\def\chapter{\addtocontents{lol}{\protect\addvspace{10pt}}\origchapter}

\makeatletter
\newcommand\myfiglisting{%
  \cleartoodd
  \chapter*{\hspace{15pt}Liste des figures}%
  \@starttoc{lof}
}
\newcommand\mytablisting{%
  \cleartoodd
  \chapter*{\hspace{15pt}Liste des tables}%
  \@starttoc{lot}
}
\newcommand\mylstlisting{%
  \cleartoodd
  \chapter*{\hspace{15pt}Sections de code}%
  \let\my@chapter\@chapter
  \renewcommand*{\@chapter}{%
  \addtocontents{lol}{\protect\addvspace{10pt}}%
  \my@chapter}
  \@starttoc{lol}
}
\makeatother

\newcommand\myalglisting{%
  \cleartoodd
  \renewcommand{\listalgorithmname}{\hspace{15pt}Liste des algorithms}
  \listofalgorithms
}

\newcommand\mylistings{%
  \myfiglisting
  \mytablisting
  \myalglisting
  \mylstlisting
}



\setlength\cftchapnumwidth{20pt}
\renewcommand{\cftchappresnum}{\hfill}
\renewcommand{\cftchapaftersnum}{\hspace*{6pt}}
\renewcommand{\cftchapfont}{\hspace{-4pt}\bf}
\setlength{\cftbeforesecskip}{.7mm}
\renewcommand\cftloftitlefont{\hspace{15pt}\bf\Huge}

% Enable the use of pseudocode
\usepackage[algochapter]{algorithm2e}
\usepackage{algorithm2e}
\usepackage{float}
% Save the "algorithm" from the algorithm2e package
\let\savedalgorithm\algorithm
\let\savedendalgorithm\endalgorithm
% Define the algorithmic environment, based on the saved environment
\newenvironment{algorithmic}{%
  \renewenvironment{algocf}[1][h]{}{}% pass over the floating stuff
  \hrule\vspace*{1.6mm}
  \savedalgorithm
}{%
  \savedendalgorithm
  \vspace*{.9mm}\hrule
}
\floatname{algorithm}{Algorithm}
% Load the algorithm package to re-define the floating environment
% "algorithm" and \listofalgorithms
\let\listofalgorithms\undefined
\usepackage[plain,chapter]{algorithm}

% Use subfloats for figures and tables (must be loaded after the caption and tocloft packages)
%\usepackage[lofdepth,lotdepth]{subfig}

% Advanced automatic reference
\usepackage{varioref}

\labelformat{chapter}{chapitre~#1}
\labelformat{section}{section~#1}
\labelformat{subsection}{section~#1}
\labelformat{subsubsection}{section~#1}
\labelformat{algorithm}{algorithme~#1}
\labelformat{figure}{figure~#1}
\labelformat{subfigure}{subfigure~\thefigure#1}
%\labelformat{table}{~#1}
\labelformat{equation}{equation~#1}
\labelformat{plot}{plot~#1}

\usepackage{chngcntr}
\counterwithin{table}{chapter}
\counterwithin{figure}{chapter}
\AtBeginDocument{\labelformat{lstlisting}{listing~#1}}
\AtBeginDocument{\counterwithin{lstlisting}{chapter}}

\DeclareCaptionFormat{listing}{{\em #1#2#3}}

\DeclareCaptionFormat{lscapelisting}{%
  \section*{\hspace{14pt}#1}
  \vspace{3mm}#3\vspace{.5mm}
}

\DeclareCaptionLabelFormat{title}{#1 #2}
\DeclareCaptionLabelFormat{hangout}{\llap{#1 #2\hspace{5mm}}}
\DeclareCaptionLabelFormat{hangoutalgo}{\llap{Algorithm #2\hspace{5mm}}}
%\captionsetup{
%	format=hang,
%	labelformat=hangout,
%	singlelinecheck=true,
%	font={footnotesize,it},
%	margin={0cm,2cm},
%}
%\captionsetup[algorithm]{
%	format=hang,
%	labelformat=hangoutalgo,
%	singlelinecheck=true,
%	font={footnotesize,it},
%	margin={0cm,2cm},
%}
%\captionsetup[subfigure]{
%	format=listing,
%	font={scriptsize},
%	singlelinecheck=true,
%	margin=0cm,
%	captionskip=3mm,
%}
%\captionsetup[lstlisting]{
%	format=hang,
%	labelformat=hangout,
%	singlelinecheck=true,
%	font={footnotesize,it},
%	margin={0cm,2cm},
%}



% Color definitions
\definecolor{highlightyellow}{RGB}{255,255,140}
\definecolor{mselogogray}{RGB}{96,101,109} % Color definition for the MSE logo
\definecolor{verylightgray}{RGB}{240,240,240}

% Enable highlighting using \HighlightFrom and \HighlightTo
\usepackage{highlighter}
\tikzset{highlighter/.style={highlightyellow, line width=0.75\baselineskip}}

% Advanced inclusion support for external documents
\usepackage{pdfpages}

% Support for conditional structures
\usepackage{ifthen}
\usepackage{xargs}
\usepackage[colorinlistoftodos,prependcaption,textsize=tiny]{todonotes}
\newcommand{\unsure}[1]{\todo[linecolor=red,backgroundcolor=red!25,bordercolor=red,inline]{TODO : #1}}
% Support for todo entries
%\newcommand{\todo}[1]{
%	{\color{red} \rule[13.5pt]{\linewidth}{.5pt}}\vspace{-13.5pt}
%	\textcolor{red}{\textbf{TODO} \hspace{3pt} #1\\[-6pt]}
%	{\color{red} \rule{\linewidth}{.5pt}\PackageWarning{TODO:}{#1!}}%
%}

\newcommand{\stodo}[1]{\reversemarginpar\hspace{0pt}\marginnote{\sffamily\scriptsize%
	{\tiny\color{gray}\textbf{TODO}\hspace{1mm}\leaders\hbox{\rule[.75mm]{1pt}{.4pt}}\hfill}\\[-.4mm]%
	#1\\[-1mm]%
	{\color{gray} \hrulefill}
}[-1.5mm]\normalmarginpar\ignorespaces
}

% Plotting support
\usepackage{pgfplots}
\usetikzlibrary{calc}
\pgfplotsset{compat=newest}

\newcommand\alignOnFrame[2]{%
    \pgfresetboundingbox
    \path ($(current axis.south west)-(#1,#2)$) rectangle 
    ($(current axis.north east)+(#1,.2cm)$);
}
\newcommandx{\improvement}[2][1=]{\todo[linecolor=Plum,backgroundcolor=Plum!25,bordercolor=Plum,#1]{#2}}


